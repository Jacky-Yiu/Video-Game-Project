\documentclass{article}
\pagestyle{empty}


\begin{document}


\section{Game}

\subsection{Description}

The game will be a 2D arcade-style game à la Pac-Man.
The user will play as an Easter Bunny, who wants to collect eggs for the
upcoming Easter.
The setting is a forest; upon collecting the necessary eggs, a portal will open
allowing the user to enter and win the game.
The Bunny also has a score, if that value falls below zero the game will be
lost.

There will be traps scattered around the forest that have various effects.
Such traps \textit{may} be hidden from view until they are activated and can do
things such as trap the Bunny for a few seconds or decrease their score.
Inversely, there will also be rewards scattered around the forest that can, for
example, add to the score or make the Bunny temporarily invulnerable.

There will also be enemies—hunters, wolves—trying to harm the Bunny.
If the Bunny runs into an enemy then the game will be lost.
Specifically, the hunter enemies also have the ability to drop traps.

The final score will be shown once the game ends and will have many factors such
as the time taken to finish, the bonus rewards collected, and more.

\subsection{Additional Notes}

Listed are some additional customizations that may be added:

\begin{itemize}
\item The maze and all its contents will be procedurally generated.
\item
    There will be several difficulty levels, which will determine everything
    from the number of required rewards to the starting health value.
\item
    The hunters will be have some sort of artificial intelligence which allows
    them to chase the Bunny in a realistic manner.
\end{itemize}


\section{Design}

The main object will be the ``game'' object, which contains the maze and the
user interface.

The ``maze'' will be represented as a 2D array of ``environment'' objects.
This class of objects will contain the walls, rewards, and traps.
The ``maze'' will also contain a list of ``character'' objects (such as the
Bunny or the enemies).
Most of the logic will be written in these ``character'' objects, as they will
handle what happens when they run over a trap or a reward.
Specifically, the Bunny object will get the current keyboard inputs and act
accordingly.

The user interface updates are called from the ``game'' object based on
information gathered from the ``maze'' object.
It will also be able to handle additional user inputs such as pausing the game.

\end{document}
